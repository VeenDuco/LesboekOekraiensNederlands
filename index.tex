% Options for packages loaded elsewhere
\PassOptionsToPackage{unicode}{hyperref}
\PassOptionsToPackage{hyphens}{url}
\PassOptionsToPackage{dvipsnames,svgnames,x11names}{xcolor}
%
\documentclass[
  letterpaper,
  DIV=11,
  numbers=noendperiod]{scrreprt}

\usepackage{amsmath,amssymb}
\usepackage{lmodern}
\usepackage{iftex}
\ifPDFTeX
  \usepackage[T1]{fontenc}
  \usepackage[utf8]{inputenc}
  \usepackage{textcomp} % provide euro and other symbols
\else % if luatex or xetex
  \usepackage{unicode-math}
  \defaultfontfeatures{Scale=MatchLowercase}
  \defaultfontfeatures[\rmfamily]{Ligatures=TeX,Scale=1}
\fi
% Use upquote if available, for straight quotes in verbatim environments
\IfFileExists{upquote.sty}{\usepackage{upquote}}{}
\IfFileExists{microtype.sty}{% use microtype if available
  \usepackage[]{microtype}
  \UseMicrotypeSet[protrusion]{basicmath} % disable protrusion for tt fonts
}{}
\makeatletter
\@ifundefined{KOMAClassName}{% if non-KOMA class
  \IfFileExists{parskip.sty}{%
    \usepackage{parskip}
  }{% else
    \setlength{\parindent}{0pt}
    \setlength{\parskip}{6pt plus 2pt minus 1pt}}
}{% if KOMA class
  \KOMAoptions{parskip=half}}
\makeatother
\usepackage{xcolor}
\setlength{\emergencystretch}{3em} % prevent overfull lines
\setcounter{secnumdepth}{5}
% Make \paragraph and \subparagraph free-standing
\ifx\paragraph\undefined\else
  \let\oldparagraph\paragraph
  \renewcommand{\paragraph}[1]{\oldparagraph{#1}\mbox{}}
\fi
\ifx\subparagraph\undefined\else
  \let\oldsubparagraph\subparagraph
  \renewcommand{\subparagraph}[1]{\oldsubparagraph{#1}\mbox{}}
\fi


\providecommand{\tightlist}{%
  \setlength{\itemsep}{0pt}\setlength{\parskip}{0pt}}\usepackage{longtable,booktabs,array}
\usepackage{calc} % for calculating minipage widths
% Correct order of tables after \paragraph or \subparagraph
\usepackage{etoolbox}
\makeatletter
\patchcmd\longtable{\par}{\if@noskipsec\mbox{}\fi\par}{}{}
\makeatother
% Allow footnotes in longtable head/foot
\IfFileExists{footnotehyper.sty}{\usepackage{footnotehyper}}{\usepackage{footnote}}
\makesavenoteenv{longtable}
\usepackage{graphicx}
\makeatletter
\def\maxwidth{\ifdim\Gin@nat@width>\linewidth\linewidth\else\Gin@nat@width\fi}
\def\maxheight{\ifdim\Gin@nat@height>\textheight\textheight\else\Gin@nat@height\fi}
\makeatother
% Scale images if necessary, so that they will not overflow the page
% margins by default, and it is still possible to overwrite the defaults
% using explicit options in \includegraphics[width, height, ...]{}
\setkeys{Gin}{width=\maxwidth,height=\maxheight,keepaspectratio}
% Set default figure placement to htbp
\makeatletter
\def\fps@figure{htbp}
\makeatother
\newlength{\cslhangindent}
\setlength{\cslhangindent}{1.5em}
\newlength{\csllabelwidth}
\setlength{\csllabelwidth}{3em}
\newlength{\cslentryspacingunit} % times entry-spacing
\setlength{\cslentryspacingunit}{\parskip}
\newenvironment{CSLReferences}[2] % #1 hanging-ident, #2 entry spacing
 {% don't indent paragraphs
  \setlength{\parindent}{0pt}
  % turn on hanging indent if param 1 is 1
  \ifodd #1
  \let\oldpar\par
  \def\par{\hangindent=\cslhangindent\oldpar}
  \fi
  % set entry spacing
  \setlength{\parskip}{#2\cslentryspacingunit}
 }%
 {}
\usepackage{calc}
\newcommand{\CSLBlock}[1]{#1\hfill\break}
\newcommand{\CSLLeftMargin}[1]{\parbox[t]{\csllabelwidth}{#1}}
\newcommand{\CSLRightInline}[1]{\parbox[t]{\linewidth - \csllabelwidth}{#1}\break}
\newcommand{\CSLIndent}[1]{\hspace{\cslhangindent}#1}

\KOMAoption{captions}{tableheading}
\makeatletter
\makeatother
\makeatletter
\@ifpackageloaded{bookmark}{}{\usepackage{bookmark}}
\makeatother
\makeatletter
\@ifpackageloaded{caption}{}{\usepackage{caption}}
\AtBeginDocument{%
\ifdefined\contentsname
  \renewcommand*\contentsname{Table of contents}
\else
  \newcommand\contentsname{Table of contents}
\fi
\ifdefined\listfigurename
  \renewcommand*\listfigurename{List of Figures}
\else
  \newcommand\listfigurename{List of Figures}
\fi
\ifdefined\listtablename
  \renewcommand*\listtablename{List of Tables}
\else
  \newcommand\listtablename{List of Tables}
\fi
\ifdefined\figurename
  \renewcommand*\figurename{Figure}
\else
  \newcommand\figurename{Figure}
\fi
\ifdefined\tablename
  \renewcommand*\tablename{Table}
\else
  \newcommand\tablename{Table}
\fi
}
\@ifpackageloaded{float}{}{\usepackage{float}}
\floatstyle{ruled}
\@ifundefined{c@chapter}{\newfloat{codelisting}{h}{lop}}{\newfloat{codelisting}{h}{lop}[chapter]}
\floatname{codelisting}{Listing}
\newcommand*\listoflistings{\listof{codelisting}{List of Listings}}
\makeatother
\makeatletter
\@ifpackageloaded{caption}{}{\usepackage{caption}}
\@ifpackageloaded{subcaption}{}{\usepackage{subcaption}}
\makeatother
\makeatletter
\@ifpackageloaded{tcolorbox}{}{\usepackage[many]{tcolorbox}}
\makeatother
\makeatletter
\@ifundefined{shadecolor}{\definecolor{shadecolor}{rgb}{.97, .97, .97}}
\makeatother
\makeatletter
\makeatother
\ifLuaTeX
  \usepackage{selnolig}  % disable illegal ligatures
\fi
\IfFileExists{bookmark.sty}{\usepackage{bookmark}}{\usepackage{hyperref}}
\IfFileExists{xurl.sty}{\usepackage{xurl}}{} % add URL line breaks if available
\urlstyle{same} % disable monospaced font for URLs
\hypersetup{
  pdftitle={Lesboek Oekraïens voor Nederlandstaligen},
  pdfauthor={Duco Veen, gebaseerd op GPT-4 output},
  colorlinks=true,
  linkcolor={blue},
  filecolor={Maroon},
  citecolor={Blue},
  urlcolor={Blue},
  pdfcreator={LaTeX via pandoc}}

\title{Lesboek Oekraïens voor Nederlandstaligen}
\author{Duco Veen, gebaseerd op GPT-4 output}
\date{Invalid Date}

\begin{document}
\maketitle
\ifdefined\Shaded\renewenvironment{Shaded}{\begin{tcolorbox}[boxrule=0pt, sharp corners, frame hidden, interior hidden, enhanced, borderline west={3pt}{0pt}{shadecolor}, breakable]}{\end{tcolorbox}}\fi

\renewcommand*\contentsname{Table of contents}
{
\hypersetup{linkcolor=}
\setcounter{tocdepth}{2}
\tableofcontents
}
\bookmarksetup{startatroot}

\hypertarget{preface}{%
\chapter*{Preface}\label{preface}}
\addcontentsline{toc}{chapter}{Preface}

\markboth{Preface}{Preface}

Kijk even in Chapter~\ref{sec-zelfstandige-naamwoorden}.

\part{inleiding-tot-de-oekraïense-taal.qmd}

Inhoud komt hier\ldots{}

\hypertarget{het-oekrauxefense-alfabet-en-de-uitspraak}{%
\chapter{Het Oekraïense alfabet en de
uitspraak}\label{het-oekrauxefense-alfabet-en-de-uitspraak}}

Het Oekraïense alfabet is gebaseerd op het Cyrillische schrift en
bestaat uit 33 letters. Het leren van het Oekraïense alfabet en de
uitspraak is een belangrijke eerste stap bij het leren van de Oekraïense
taal. Hieronder vind je een tabel met het Oekraïense alfabet, de
overeenkomstige Nederlandse benamingen en de uitspraak van elke letter.

\begin{longtable}[]{@{}
  >{\raggedright\arraybackslash}p{(\columnwidth - 4\tabcolsep) * \real{0.2881}}
  >{\raggedright\arraybackslash}p{(\columnwidth - 4\tabcolsep) * \real{0.3390}}
  >{\raggedright\arraybackslash}p{(\columnwidth - 4\tabcolsep) * \real{0.3729}}@{}}
\toprule()
\begin{minipage}[b]{\linewidth}\raggedright
Oekraïense letter
\end{minipage} & \begin{minipage}[b]{\linewidth}\raggedright
Nederlandse benaming
\end{minipage} & \begin{minipage}[b]{\linewidth}\raggedright
Uitspraak (benadering)
\end{minipage} \\
\midrule()
\endhead
А, а & A & a (als in ``kat'') \\
Б, б & Be & b \\
В, в & Ve & v \\
Г, г & Ge & h of g (zacht) \\
Ґ, ґ & Ge met een opzetstreepje & g (hard) \\
Д, д & De & d \\
Е, е & Je & je (als in ``jenever'') \\
Є, є & Je met een opzetstreepje & je (als in ``jeugd'') \\
Ж, ж & Zje & zh (als in ``vision'') \\
З, з & Ze & z \\
И, и & I & i (als in ``vis'') \\
І, і & I met een opzetstreepje & i (als in ``blik'') \\
Ї, ї & Ji met een opzetstreepje & yi (als in ``yoga'') \\
Й, й & Jot & y (als in ``yes'') \\
К, к & Ka & k \\
Л, л & El & l \\
М, м & Em & m \\
Н, н & En & n \\
О, о & O & o (als in ``bot'') \\
П, п & Pe & p \\
Р, р & Er & r \\
С, с & Es & s \\
Т, т & Te & t \\
У, у & Oe & oe (als in ``voet'') \\
Ф, ф & Ef & f \\
Х, х & Cha & kh (als in ``Bach'') \\
Ц, ц & Tse & ts \\
Ч, ч & Tsje & ch (als in ``check'') \\
Ш, ш & Sja & sh (als in ``shop'') \\
Щ, щ & Sjtsja & shch (als in ``fresh cheese'') \\
Ь, ь & Zachte teken & - (verzacht de voorafgaande medeklinker) \\
Ю, ю & Joo & yoe (als in ``Joep'') \\
Я, я & Ja & ya (als in ``yacht'') \\
\bottomrule()
\end{longtable}

De uitspraak in deze tabel is slechts een benadering en dient als een
basis voor beginners. De precieze uitspraak kan variëren.

Om verder te oefenen met de uitspraak vind je hieronder een tabel met
simpele zinnetjes. Deze zinnen hebben woorden die allemaal met dezelfde
letter beginnen (op wat voorzetsels na). Je krijgt in de tweede kolom de
uitspraak te lezen en in de derde kolom de Nederlandse vertaling. Veel
plezier met oefenen!

\begin{longtable}[]{@{}
  >{\raggedright\arraybackslash}p{(\columnwidth - 4\tabcolsep) * \real{0.3333}}
  >{\raggedright\arraybackslash}p{(\columnwidth - 4\tabcolsep) * \real{0.3333}}
  >{\raggedright\arraybackslash}p{(\columnwidth - 4\tabcolsep) * \real{0.3333}}@{}}
\toprule()
\begin{minipage}[b]{\linewidth}\raggedright
Zin
\end{minipage} & \begin{minipage}[b]{\linewidth}\raggedright
Uitspraak
\end{minipage} & \begin{minipage}[b]{\linewidth}\raggedright
Nederlandse vertaling
\end{minipage} \\
\midrule()
\endhead
Ананас апетитно акуратно в апельсин & ananas apetytno akuratno v apelsin
& Ananas smakelijk en netjes in sinaasappel \\
Бабуся бачить білку під березою & babusya bachyt bilku pid berezoyu &
Oma ziet eekhoorn onder de berk \\
Ведмідь великий відвідує вербу & vedmid velykyi vidviduye verbu & Grote
beer bezoekt wilg \\
Горобець грається на гілці гарно & horobets hrayetsya na hiltci harno &
Mus speelt mooi op tak \\
Ґава з ґудзиком під ґанок ґрати & hava z hudzykom pid hanok graty &
Kloof met knoop onder veranda spelen \\
Дитина дивиться на дельфіна далеко & dytyna dyvytsya na delfina daleko &
Kind kijkt ver naar dolfijn \\
Жираф жує жолудь з жартом & zhyraf zhuye zholud z zhartom & Giraf kauwt
op eikel met grap \\
Зайчик забавляє зебру біля замку & zaychyk zabavlyaye zebru bilya zamku
& Haasje vermaakt zebra bij kasteel \\
Їжачок їсть їжу з їдким смаком & yizhachok yist yizhu z yidkym smakom &
Egel eet scherp eten met scherpe smaak \\
Йогурт йосипа йод з йомого йойка & yohurt yosypa yod z yomoho yoyka &
Yoghurt jodium van Jozef vanavond zingen \\
Качка купається в калюжі біля квітки & kachka kupayetsya v kalyuzhi
bilya kvitky & Eend baadt in plas bij bloemen \\
Лис ловить лелеку легко в лісі & lys lovyt leleku lehko v lisi & Vos
vangt ooievaar gemakkelijk in bos \\
Мавка малює місяць мило на марній & mavka malyuye misyats mylo na marniy
& Nimf schildert maan lief en tevergeefs \\
Носоріг несе нектарини на носі & nosorih nese nektaryny na nosi &
Neushoorn draagt nectarines op zijn neus \\
Олень одягає окуляри охайно в офісі & olen od'yahaye okulary ohayno v
ofisi & Hert zet netjes bril op kantoor \\
Папуга повторює пісні привітно з птахами & papuha povtoryuye pisni
pryvitno z ptakhamy & Papegaai herhaalt liedjes vriendelijk met
vogels \\
Риба регоче радість ритмічно в річці & ryba rehoche radist rytmichno v
richci & Vis lacht vrolijk ritmisch in rivier \\
Слон сміється на сонці світло й смачно & slon smiyetsya na sontsi svitlo
y smachno & Olifant lacht zonnig licht en lekker \\
Тигр танцює тільки тихо біля троянд & tyr tanʹtsʹuye tikho bilya troyand
& Tijger danst enkel stil bij rozen \\
Улитка усміхається уночі уявляючи щось & ulitka usmikhaetsya unochi
uayvlyayuchy shchos & Slak glimlacht 's nachts, iets inbeeldend \\
Фея фотографує фрукти на фоні фіалок & feya fotohrafuye frukty na foni
fialok & Fee fotografeert fruit op achtergrond van viooltjes \\
Хлопчик хоче хліба хрусткого на хмаринці & khlopchyk hoche khliba
khrustkoho na hmaryntsi & Jongen wil knapperig brood op wolk \\
Цирк цікаво цокотить на цементі & tsyrk tsikavo tsokotyt na tsementi &
Circus klikt interessant op cement \\
Чоловік чистить чоботи чемно під часом & cholovik chystyt choboty chemno
pid chasom & Man poetst laarzen zorgvuldig tijdens de tijd \\
Школяр шукає шлях до школи швидко & shkolyar shukaye shlyakh do shkoly
shvydko & Scholier zoekt snel weg naar school \\
Щасливий щук щебече щедро на щодині & shchaslyvy shchuk shchebeche
shedro na shchodini & Gelukkige snoek tjilpt gul elk uur \\
Юний юрбаніст юродує в юрти & yunyi yurbanist yuroduye v yurti & Jonge
stedenbouwkundige verheugt zich in yurt \\
Ясен якісно яскраво й ярко на ярмарку & yasen yakisno yaskravo y yarko
na yarmarku & Es kwalitatief helder en fel op de markt \\
\bottomrule()
\end{longtable}

\part{Basisgrammatica.qmd}

Inhoud komt hier\ldots{}

\hypertarget{sec-zelfstandige-naamwoorden}{%
\chapter{Zelfstandige Naamwoorden}\label{sec-zelfstandige-naamwoorden}}

In dit hoofdstuk behandelen we zelfstandige naamwoorden in het
Oekraïens. We zullen ingaan op geslacht, naamvallen, meervoudsvormen en
enkele belangrijke regels met betrekking tot zelfstandige naamwoorden.

Een zelfstandig naamwoord is een woord dat wordt gebruikt om een
persoon, dier, plaats, object, idee of gebeurtenis te benoemen. In zowel
het Nederlands als het Oekraïens zijn zelfstandige naamwoorden
essentieel in de grammaticale structuur van zinnen, omdat ze fungeren
als het onderwerp of object van een werkwoord. Zelfstandige naamwoorden
zijn er in verschillende geslachten (mannelijk, vrouwelijk en onzijdig)
en worden meestal beïnvloed door verbuigingen op basis van de naamval
waarin ze voorkomen.

\hypertarget{geslacht-van-zelfstandige-naamwoorden}{%
\section{Geslacht van zelfstandige
naamwoorden}\label{geslacht-van-zelfstandige-naamwoorden}}

Oekraïense zelfstandige naamwoorden hebben drie geslachten: mannelijk,
vrouwelijk en onzijdig. In dit gedeelte leer je hoe je het geslacht van
een zelfstandig naamwoord kunt herkennen.

\hypertarget{mannelijke-zelfstandige-naamwoorden}{%
\subsection{Mannelijke zelfstandige
naamwoorden}\label{mannelijke-zelfstandige-naamwoorden}}

Mannelijke zelfstandige naamwoorden vormen een belangrijke groep in de
Oekraïense taal. Ze worden vaak gebruikt om personen, dieren, objecten
en concepten aan te duiden. Het herkennen van mannelijke zelfstandige
naamwoorden is essentieel om de juiste grammaticale regels toe te
passen, zoals verbuiging en overeenstemming met bijvoeglijke
naamwoorden. Hier zijn enkele richtlijnen om mannelijke zelfstandige
naamwoorden in het Oekraïens te identificeren:

\begin{enumerate}
\def\labelenumi{\arabic{enumi}.}
\item
  \textbf{Eindletter}: Mannelijke zelfstandige naamwoorden eindigen
  meestal op een medeklinker. Enkele veelvoorkomende eindletters zijn
  -т, -н, -р, -л, -к, -й, -ь en -й. Er zijn echter uitzonderingen op
  deze regel, dus het is belangrijk om ook naar andere kenmerken te
  kijken.
\item
  \textbf{Verbuiging}: Mannelijke zelfstandige naamwoorden volgen
  specifieke verbuigingspatronen in verschillende naamvallen. Als je
  merkt dat een zelfstandig naamwoord een mannelijk verbuigingspatroon
  volgt, is het hoogstwaarschijnlijk een mannelijk zelfstandig
  naamwoord.
\item
  \textbf{Woordbetekenis}: In sommige gevallen kan de betekenis van het
  woord je helpen het geslacht te bepalen. Mannelijke zelfstandige
  naamwoorden verwijzen vaak naar mannelijke personen of dieren, zoals
  батько (vader) en кіт (kater).
\item
  \textbf{Overeenstemming met bijvoeglijke naamwoorden}: Mannelijke
  zelfstandige naamwoorden stemmen overeen met mannelijke bijvoeglijke
  naamwoorden. Als je merkt dat een zelfstandig naamwoord wordt
  voorafgegaan of gevolgd door een mannelijk bijvoeglijk naamwoord, kun
  je er vrij zeker van zijn dat het zelfstandig naamwoord mannelijk is.
\end{enumerate}

Het is belangrijk op te merken dat er altijd uitzonderingen zijn op de
bovengenoemde regels. Hoewel deze richtlijnen je kunnen helpen bij het
identificeren van veel mannelijke zelfstandige naamwoorden, is het
raadzaam om bekend te raken met de woordenschat en de verschillende
verbuigingen van zelfstandige naamwoorden om nauwkeuriger te zijn in je
identificatie.

Enkele voorbeelden van mannelijke zelfstandige naamwoorden in het
Oekraïens zijn:

\begin{itemize}
\tightlist
\item
  дім (huis)
\item
  стіл (tafel)
\item
  город (tuin)
\item
  учитель (leraar)
\item
  книга (boek)
\end{itemize}

Nu je weet hoe je mannelijke zelfstandige naamwoorden kunt herkennen,
kun je verder gaan met het leren van andere kenmerken van zelfstandige
naamwoorden in het Oekraïens, zoals vrouwelijke en onzijdige
zelfstandige naamwoorden.

\hypertarget{vrouwelijke-zelfstandige-naamwoorden}{%
\subsection{Vrouwelijke zelfstandige
naamwoorden}\label{vrouwelijke-zelfstandige-naamwoorden}}

Vrouwelijke zelfstandige naamwoorden vormen een andere belangrijke groep
in de Oekraïense taal. Ze worden vaak gebruikt om vrouwelijke personen,
dieren, objecten en concepten aan te duiden. Het herkennen van
vrouwelijke zelfstandige naamwoorden is essentieel om de juiste
grammaticale regels toe te passen, zoals verbuiging en overeenstemming
met bijvoeglijke naamwoorden. Hier zijn enkele richtlijnen om
vrouwelijke zelfstandige naamwoorden in het Oekraïens te identificeren:

\begin{enumerate}
\def\labelenumi{\arabic{enumi}.}
\item
  \textbf{Eindletter}: Vrouwelijke zelfstandige naamwoorden eindigen
  meestal op -а, -я, -ь of -і. Deze eindletters komen vaak voor bij
  vrouwelijke zelfstandige naamwoorden, hoewel er enkele uitzonderingen
  zijn.
\item
  \textbf{Verbuiging}: Vrouwelijke zelfstandige naamwoorden volgen
  specifieke verbuigingspatronen in verschillende naamvallen. Als je
  merkt dat een zelfstandig naamwoord een vrouwelijk verbuigingspatroon
  volgt, is het hoogstwaarschijnlijk een vrouwelijk zelfstandig
  naamwoord.
\item
  \textbf{Woordbetekenis}: In sommige gevallen kan de betekenis van het
  woord je helpen het geslacht te bepalen. Vrouwelijke zelfstandige
  naamwoorden verwijzen vaak naar vrouwelijke personen of dieren, zoals
  мама (moeder) en кішка (kat).
\item
  \textbf{Overeenstemming met bijvoeglijke naamwoorden}: Vrouwelijke
  zelfstandige naamwoorden stemmen overeen met vrouwelijke bijvoeglijke
  naamwoorden. Als je merkt dat een zelfstandig naamwoord wordt
  voorafgegaan of gevolgd door een vrouwelijk bijvoeglijk naamwoord, kun
  je er vrij zeker van zijn dat het zelfstandig naamwoord vrouwelijk is.
\end{enumerate}

Het is belangrijk op te merken dat er altijd uitzonderingen zijn op de
bovengenoemde regels. Hoewel deze richtlijnen je kunnen helpen bij het
identificeren van veel vrouwelijke zelfstandige naamwoorden, is het
raadzaam om bekend te raken met de woordenschat en de verschillende
verbuigingen van zelfstandige naamwoorden om nauwkeuriger te zijn in je
identificatie.

Enkele voorbeelden van vrouwelijke zelfstandige naamwoorden in het
Oekraïens zijn:

\begin{itemize}
\tightlist
\item
  країна (land)
\item
  кімната (kamer)
\item
  людина (persoon)
\item
  дорога (weg)
\item
  річка (rivier)
\end{itemize}

Nu je weet hoe je vrouwelijke zelfstandige naamwoorden kunt herkennen,
kun je verder gaan met het leren van het volgende kenmerk van
zelfstandige naamwoorden in het Oekraïens, onzijdige zelfstandige
naamwoorden.

\hypertarget{onzijdige-zelfstandige-naamwoorden}{%
\subsection{Onzijdige zelfstandige
naamwoorden}\label{onzijdige-zelfstandige-naamwoorden}}

Onzijdige zelfstandige naamwoorden vormen de derde belangrijke groep in
de Oekraïense taal. Ze worden gebruikt om objecten, dieren, concepten en
ideeën aan te duiden die noch mannelijk noch vrouwelijk zijn. Hier zijn
enkele richtlijnen om onzijdige zelfstandige naamwoorden in het
Oekraïens te identificeren:

\begin{enumerate}
\def\labelenumi{\arabic{enumi}.}
\item
  \textbf{Eindletter}: Onzijdige zelfstandige naamwoorden eindigen
  meestal op -о, -е, of -я. Deze eindletters komen vaak voor bij
  onzijdige zelfstandige naamwoorden, hoewel er enkele uitzonderingen
  zijn.
\item
  \textbf{Verbuiging}: Ook onzijdige zelfstandige naamwoorden volgen
  specifieke verbuigingspatronen in verschillende naamvallen. Als je
  merkt dat een zelfstandig naamwoord een onzijdig verbuigingspatroon
  volgt, is het hoogstwaarschijnlijk een onzijdig zelfstandig naamwoord.
\item
  \textbf{Woordbetekenis}: In sommige gevallen kan de betekenis van het
  woord je helpen het geslacht te bepalen. Onzijdige zelfstandige
  naamwoorden verwijzen vaak naar objecten, dieren of concepten zonder
  een specifiek geslacht, zoals вікно (raam) en місто (stad).
\item
  \textbf{Overeenstemming met bijvoeglijke naamwoorden}: Onzijdige
  zelfstandige naamwoorden stemmen overeen met onzijdige bijvoeglijke
  naamwoorden. Als je merkt dat een zelfstandig naamwoord wordt
  voorafgegaan of gevolgd door een onzijdig bijvoeglijk naamwoord, kun
  je er vrij zeker van zijn dat het zelfstandig naamwoord onzijdig is.
\end{enumerate}

Het is belangrijk op te merken dat er altijd uitzonderingen zijn op de
bovengenoemde regels. Hoewel deze richtlijnen je kunnen helpen bij het
identificeren van veel onzijdige zelfstandige naamwoorden, is het
raadzaam om bekend te raken met de woordenschat en de verschillende
verbuigingen van zelfstandige naamwoorden om nauwkeuriger te zijn in je
identificatie.

Enkele voorbeelden van onzijdige zelfstandige naamwoorden in het
Oekraïens zijn:

\begin{itemize}
\tightlist
\item
  дерево (boom)
\item
  море (zee)
\item
  село (dorp)
\item
  поле (veld)
\item
  кільце (ring)
\end{itemize}

Nu je weet hoe je onzijdige zelfstandige naamwoorden kunt herkennen, kun
je verder gaan met het leren van andere kenmerken van zelfstandige
naamwoorden in het Oekraïens, zoals naamvallen en verbuiging.

\hypertarget{naamvallen}{%
\section{Naamvallen}\label{naamvallen}}

In het Oekraïens zijn er zeven naamvallen die de grammaticale rol van
een zelfstandig naamwoord in de zin bepalen. In dit gedeelte worden de
naamvallen en hun functies besproken.

\hypertarget{nominatief}{%
\subsection{Nominatief}\label{nominatief}}

(Tekst over de nominatief en het gebruik ervan.)

\hypertarget{genitief}{%
\subsection{Genitief}\label{genitief}}

(Tekst over de genitief en het gebruik ervan.)

\hypertarget{datief}{%
\subsection{Datief}\label{datief}}

(Tekst over de datief en het gebruik ervan.)

\hypertarget{accusatief}{%
\subsection{Accusatief}\label{accusatief}}

(Tekst over de accusatief en het gebruik ervan.)

\hypertarget{instrumentalis}{%
\subsection{Instrumentalis}\label{instrumentalis}}

(Tekst over de instrumentalis en het gebruik ervan.)

\hypertarget{locatief}{%
\subsection{Locatief}\label{locatief}}

(Tekst over de locatief en het gebruik ervan.)

\hypertarget{vocatief}{%
\subsection{Vocatief}\label{vocatief}}

(Tekst over de vocatief en het gebruik ervan.)

\hypertarget{meervoud-van-zelfstandige-naamwoorden}{%
\section{Meervoud van zelfstandige
naamwoorden}\label{meervoud-van-zelfstandige-naamwoorden}}

In dit gedeelte leer je hoe je het meervoud van zelfstandige naamwoorden
in het Oekraïens kunt vormen.

\hypertarget{reguliere-meervoudsvorming}{%
\subsection{Reguliere
meervoudsvorming}\label{reguliere-meervoudsvorming}}

(Tekst over de reguliere manier om meervoudsvormen te maken.)

\hypertarget{onregelmatige-meervoudsvorming}{%
\subsection{Onregelmatige
meervoudsvorming}\label{onregelmatige-meervoudsvorming}}

(Tekst over onregelmatige meervoudsvormen en uitzonderingen.)

\hypertarget{belangrijke-regels-en-uitzonderingen}{%
\section{Belangrijke regels en
uitzonderingen}\label{belangrijke-regels-en-uitzonderingen}}

In dit gedeelte worden enkele belangrijke regels en uitzonderingen met
betrekking tot Oekraïense zelfstandige naamwoorden besproken.

(Tekst over belangrijke regels en uitzonderingen.)

\hypertarget{oefeningen}{%
\section{Oefeningen}\label{oefeningen}}

In dit gedeelte vind je oefeningen om je kennis van Oekraïense
zelfstandige naamwoorden te testen en te oefenen.

(Oefeningen voor geslachtsherkenning, meervoudsvorming, naamvallen,
enz.)

\hypertarget{bijvoeglijke-naamwoorden.qmd}{%
\chapter{bijvoeglijke-naamwoorden.qmd}\label{bijvoeglijke-naamwoorden.qmd}}

Inhoud komt hier\ldots{}

\hypertarget{werkwoorden.qmd}{%
\chapter{werkwoorden.qmd}\label{werkwoorden.qmd}}

Inhoud komt hier\ldots{}

\hypertarget{telwoorden.qmd}{%
\chapter{telwoorden.qmd}\label{telwoorden.qmd}}

Inhoud komt hier\ldots{}

\hypertarget{voorzetsels-en-bijwoorden.qmd}{%
\chapter{voorzetsels-en-bijwoorden.qmd}\label{voorzetsels-en-bijwoorden.qmd}}

Inhoud komt hier\ldots{}

\part{Appendixes.qmd}

Inhoud komt hier\ldots{}

\hypertarget{references}{%
\chapter*{References}\label{references}}
\addcontentsline{toc}{chapter}{References}

\markboth{References}{References}

\hypertarget{refs}{}
\begin{CSLReferences}{0}{0}
\end{CSLReferences}



\end{document}
